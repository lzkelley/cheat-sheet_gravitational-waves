\documentclass[10pt, oneside, onecolumn]{article}   	% use "amsart" instead of "article" for AMSLaTeX format
\usepackage[margin=1in]{geometry}                		% See geometry.pdf to learn the layout options. There are lots.
\geometry{letterpaper}                   		% ... or a4paper or a5paper or ...
\usepackage{graphicx}				% Use pdf, png, jpg, or eps§ with pdflatex; use eps in DVI mode
% TeX will automatically convert eps --> pdf in pdflatex
\usepackage{amssymb}
\usepackage{natbib}
\usepackage{hyperref}

\usepackage{color}
% \usepackage{booktabs}   % Fancier tables
\usepackage{tabularx}
\usepackage{float}
\usepackage{totcount}
\usepackage{cuted}

\usepackage[dvipsnames, table]{xcolor} % adds colors to color space (https://en.wikibooks.org/wiki/LaTeX/Colors)
\usepackage{graphicx}
\usepackage{multirow}
\usepackage[section]{placeins}   % Force figures to stay in their sections (tex.stackexchange.com/q/279/22806)
\usepackage{subfig}
\usepackage{widetext}
\usepackage{upgreek}    % allow alternative greek letters like $\uptau$
\usepackage{enumitem}   % customize enumerate symbols
\setlist[enumerate]{leftmargin=*}   % Set left-margin of enumerate lists to match the edge
\setlist[itemize]{leftmargin=*}   % Set left-margin of enumerate lists to match the edge


\usepackage{lipsum} % generate dummy text
\usepackage{amsmath}
\usepackage[thinc]{esdiff}  % easier/convenient derivatives and partial derivatives
\usepackage{amssymb}
\usepackage{subfiles}    % be able to compile the included sub-files
\usepackage[explicit, calcwidth]{titlesec}
\usepackage[normalem]{ulem}


% \input{AstroComs}                   % Common Astro Commands
% \input{CustComs}                   % Common Astro Commands
\input{jabbrevs}                        % Journal Abbreviations
\input{coms}                        % Journal Abbreviations

% \newcommand{\hc}{h_\trt{c}}
% \newcommand{\hs}{h_\trt{s}}
\newcommand{\hsn}{h_\tr{s,n}}
\newcommand{\hscirc}{h_\trt{s,circ}}
\newcommand{\hcs}{h_\trt{c,s}}

\newcommand{\frst}{f_\tr{r}}
\newcommand{\frstgw}{f_{\trt{GW},r}}
\newcommand{\frstorb}{f_{\trt{orb},r}}
\newcommand{\fobsorb}{f_{\trt{orb},o}}
\newcommand{\fobsgw}{f_{\trt{GW},o}}
\newcommand{\frstn}{f_\trt{n,r}}
\newcommand{\fobsn}{f_\trt{n,o}}

\newcommand\erfc[1]{\mathrm{erfc}\left(#1\right)}
\newcommand\erfcinv[1]{\mathrm{erfc}^{-1}\left(#1\right)}

\newcommand{\lgw}{L_\trt{GW}}
\newcommand{\lgwn}{L_{\trt{GW},n}}
\newcommand{\lgwc}{L_\trt{GW,circ}}
\newcommand{\egw}{\varepsilon_\trt{GW}}   % energy in GW
% \newcommand{\thard}{\bigt_\tr{h}}
% \newcommand{\thardi}{\bigt_{\tr{h},i}}
% \newcommand{\thardij}{\bigt_{\tr{h},ij}}
\newcommand{\tdur}{T_\trt{dur}}

% == Terminology Shortcuts ==
%\newcommand{\mbhb}{MBHB}  % {MBH-Binaries}

\newcommand{\frest}{f_\trt{r}}
\newcommand{\fobs}{f_\trt{o}}
\newcommand{\fharm}{f_\trt{h}}

\newcommand{\thard}{\tau_\trt{r}}
\newcommand{\tgw}{\tau_\trt{gw}}
% \newcommand{\tgwc}{\bigt_\tr{GW,circ}}
\newcommand{\tgwc}{\tau_\trt{gw,circ}}


\newcommand\mred{\mu_\trt{red}}

\newcommand\snr{\mathcal{R}}
\newcommand\snra{\mbox{$\mathcal{R}_A$}}
\newcommand\snrb{\mbox{$\mathcal{R}_B$}}
\newcommand\snrta{\mbox{$\mathcal{R}_A^T$}}
\newcommand\snrtb{\mbox{$\mathcal{R}_B^T$}}
\newcommand\pulsarsum{\sum_k \sum_{ij}}
\newcommand\Gij{\Gamma_{ij}}


\titleformat{\section}{\normalfont\Large\bfseries\color{Red}}{\thesubsection}{1em}{#1}[{\titlerule[0.8pt]}]
\titleformat{\subsection}[block]{\large\bfseries}{\thetitle}{.5em}{#1}[\vspace{-8pt}\rule{\titlewidth}{1pt}]

\newlength\tindent
\setlength{\tindent}{\parindent}
\setlength{\parindent}{0pt}
\renewcommand{\indent}{\hspace*{\tindent}}



\begin{document}

    \tableofcontents

    % ===========================
    % ====    Definitions    ====
    % ===========================

    \section{Definitions}

        \begin{table}[h]
        \centering
        \begin{tabular}{@{} lcrc @{}} % Column formatting, @{} suppresses leading/trailing space
            % \toprule
            Name			& Symbol	& Description	& Equation \\ \hline
            Total Mass 		& $M$		& Sum of individual masses: $M = m_1 + m_2$	& [\refeq{eq:chirp_mass}] \\
            Mass Ratio		& $q$		& $q \equiv m_2 / m_1 \msp \mathrm{s.t.} \msp q \leq 1 \msp \& \msp m_2 < m_1$ & [\refeq{eq:chirp_mass}] \\
            Chirp Mass		& $\mchirp$	& Characteristic GW mass 					& [\refeq{eq:chirp_mass}] \\
            Reduced Mass	& $\mred$	& $\mred \equiv m_1 m_2 / (m_1 + m_2) = q/(1+q)$ 		& [\refeq{eq:chirp_mass}] \\ \hline
            Semi-major axis & $a$		& Binary separation (rest frame), $a^3 / GM = (T / 2\pi)^2$. &  \\ \hline
            r[est-frame]	& $X_{r}$	& Quantities evaluated in the source's rest-frame 	&	 	\\
            o[bserver-frame]	& $X_{o}$	& Quantities evaluated in the observer's frame 	& 	\\

            &				&												&	\\ \hline
        \end{tabular}
        \end{table}

        \subsection{Binaries \& Orbits}
            We adopt the convention that the \textbf{Chirp-Mass} is an intrinsic, invariant property of the binary (i.e.~it is not redshifted):
            \begin{equation}
                \label{eq:chirp_mass}
                \mchirp = \frac{\left(m_1 m_2\right)^{3/5}}{M^{1/5}} = M \frac{q^{3/5}}{\left(1 + q\right)^{6/5}} = M^{2/5} \mred^{3/5}.
            \end{equation}
            Recall Kepler's law,
            \begin{equation}
                \frac{a^3}{GM} = \scale[2]{T}{2\pi} = \lr[-2]{2\pi \frstorb}.
            \end{equation}

            The GW \textbf{eccentricity function} is,
            \begin{equation}
                \label{eq:}
                F(e) \equiv \frac{1 + \frac{73}{24} e^2 + \frac{37}{96} e^4}{\left( 1 - e^2\right)^{7/2}} = \sum_{n=1}^\infty g(n,e),
            \end{equation}
            and the GW \textbf{frequency distribution function} can be expressed as \citep[][Eq.~2.4]{enoki2007a},
            \begin{equation}
            \begin{split}
                \label{eq:freq_dist_func}
                g(n,e) \equiv \frac{n^4}{32} \Biggl(& \left[ J_{n-2}(ne) - 2eJ_{n-1}(ne) + \frac{2}{n} J_n (ne) + 2e J_{n+1} (ne) - J_{n+2} (ne) \vphantom{\frac{2}{n}}\right]^2 \\
                &+ \left(1-e^2\right) \Bigl[J_{n-2}(ne) - 2eJ_n(ne) + J_{n+2}(ne)\Bigr]^2 + \frac{4}{3n^2}\Bigl[J_n(ne)\Bigr]^2 \Biggr),
            \end{split}
            \end{equation}
            where $J_n(x)$ is the $n$th-order Bessel function of the first kind.  Note that for zero eccentricity, $g(n=2,0) = 1$ and $g(n\neq2, 0) = 0$.

        % ====    Power Spectral Density    ====
        \subsection{(Power) Spectral (Energy) Density \NOTE{NEEDS REVIEW}}
            Noise \textbf{Power Spectral Density (PSD)} $S_n(f)$ has units of inverse frequency.  PSD is `one-sided' because the signal is real, and thus integrals over all frequencies can be replaced by ones over positive-frequency only (i.e.~$S_n(f) = S_n(-f)$).  PSD integrated over all positive frequencies gives the mean-square noise amplitude \citep{moore2015}.
            \begin{equation}
                \overline{|n(t)|^2} = \int_0^\infty S_n(f) \, df
            \end{equation}
            For source amplitudes \citep{rosado2015} (\textit{\textbf{NOTE:} the power-spectral-density is different for LIGO vs.~PTA}),
            \begin{equation}
                S_h(f) = \frac{h_c^2}{12 \pi^2 f^3}
            \end{equation}

            Extracting a signal involves constructing a `Wiener optimal filter'\footnote{The `filter' is an approximation to the input signal ($\hat{h}(t)$ approximating, in this case, $h(t)$), where it is assumed that the optimal filter can be expressed as a linear function of the measured signal, i.e. $\hat{h}_k = \sum_i l_{k-i} s_i$, and thus the coefficients $l_k$ must be solved for.  The solution is found by minimizing the error between the estimate and the true signal.}

            % =============   Energy Density   ============
            % \subsection{Energy Density}
            \textbf{Energy Density} carried by a GW (averaged over several wavelengths/periods) is \citep{moore2015},
            \begin{equation}
                \rho c^2 = \frac{c^2}{16\pi G} \int_0^\infty \frac{\pi c^2}{4G} f^2 \, S_h(f) \, df.
            \end{equation}
            The \textbf{Spectral Energy Density} is defined as the integrand of the above, which is the energy per unit volume of space, per unit frequency \citep{moore2015},
            \begin{equation}
                S_E(f) = \frac{\pi c^2}{4 G} f^2 \, S_h(f);
            \end{equation}
            and an analogous expression for the noise simply replaces $S_h$ with $S_n$.  This is related to the fractional energy-density per logarithmic frequency interval as,
            \begin{equation}
                \Omega_{GW}(f) = \frac{f \, S_E(f)}{\rho_c c^2},
            \end{equation}
            where the cosmological critical density is,
            \begin{equation}
                \rho_c = \frac{3 H_0^2}{8\pi G}.
            \end{equation}

            \begin{equation*}
                H_0^2 \, \Omega_{GW}(f) = \frac{8\pi G}{3 c^2} f \, S_E(f) = \frac{2 \pi^2}{3} \, f^3 \, S_h(f) = \frac{2\pi^2}{3} f^2 \left[h_c(f) \right]^2 = \frac{8\pi^2}{3} f^4 \left| \tilde{h}(f)\right|^2.
            \end{equation*}



    % ==========================
    % ====    GW Sources    ====
    % ==========================

    \section{Gravitational Wave Sources}

        \subsection{Time and Hardening Rates}

            The hardening rate for a circular binary is \citep[][Eq.5.6]{Peters1964},
            \begin{equation}
                \label{eq:gw_hard}
                \frac{da}{dt} = -\frac{64 \, G^3}{5 \, c^5} \frac{M_1 \, M_2 \, M}{a^3} F(e).
            \end{equation}

            The total time to go between separations $a_1$ and $a_2$ is then,
            \begin{align}
                \label{eq:gw_coal_time}
                t & = \frac{5 \, c^5}{256 \, G^3} \frac{a_1^4 - a_2^4}{M_1 \, M_2 \, M} F(e)^{-1}, \\
                & \approx 0.2 \, \rm{Gyr} \scale[4]{a}{0.1 \, \rm{pc}} \scale[3]{M}{10^9 \, \msol} \scale{1/q + 2 + q}{4}, \\
            \end{align}

            Following \citep{enoki2007a}, the standard hardening timescale is defined with respect to frequency, i.e.,
            \begin{equation}
                \thard \equiv \frac{d t_r}{d\ln f_r} = \frac{f_r}{df_r/dt_r} = - \frac{2}{3} \frac{a}{da/dt_r}.
            \end{equation}
            Note that this quantity does vary between the observer-frame and rest-frame, s.t. $\tau_o = \tau_r \cdot \lr{1+z}$.

            The GW-hardening timescale (in the rest-frame) is,
            \begin{align}
                \label{eq:time_hard_gw}
                \tgw & = \frac{\tgwc}{F(e)} \\
                \tgwc & = \frac{5}{96}\left(\frac{G\mchirp}{c^3}\right)^{-5/3} \left(2 \pi \frstorb \right)^{-8/3}
            \end{align}

        \subsection{Energy, Power, and Strain}

            GW power is emitted at integer harmonics of the orbital frequency, $\frstn = n \cdot \frstorb$, which is also redshifted to reach the observer-frame, $\fobsn = n \cdot \frstorb / (1+z)$.  The energy emitted in GWs, per interval of rest-frame frequency, integrated over binary lifetime, is \citet[][Eq.~3.10]{enoki2007a},
            \begin{subequations}
            \begin{align}
                \label{eq:gw_energy_spectrum}
                \frac{d \egw}{d \frst} & = \int \lgwc(f_p) \, \frac{dt_r}{df_p} \, \sum_{n=1}^\infty \, g(n,e) \, \delta(\frst - n f_p) \, d f_p \\
                    & = \sum_{n=1}^\infty \left[ \lgwc(f_p) \cdot \frac{\thard(f_p)}{n f_p} \cdot g(n,e) \right]_{f_p = \frst/n} \\
                    & = \sum_{n=1}^\infty \left[ \lgwn(f_p) \cdot \frac{\thard(f_p)}{n f_p} \right]_{f_p = \frst/n},
            \end{align}
            \end{subequations}
            where the power-radiated at each harmonic, in the rest-frame, is:
            \begin{equation}
                \lgwn = \lgwc \cdot g(n,e).
            \end{equation}

            Total GW power emitted is the sum over harmonics \citep[][Eq.~2.2]{enoki2007a},
            \begin{align}
                \label{eq:lum_gw}
                \lgw & = \sum_n \lgwn = \lgwc \sum_n g(n,e) = \lgwc \cdot F(e), \\
                \lgwc & = \frac{32}{5 G c^5} \left(G\mchirp\right)^{10/3} \left( 2\pi \frstorb \right)^{10/3}.
            \end{align}
            Such that, for circular binaries,
            \begin{subequations}
            \begin{align}
                \frac{d \egw}{d \frst} \bigg|_{\tr{circ}} & = \frac{\pi}{3 G} \lr[5/3]{G \mchirp} \lr[-1/3]{\pi \frst}.
            \end{align}
            \end{subequations}

            The GW strain amplitude at a given harmonic can be related to the (rest-frame) luminosity as \citep[][Eq.~2.1]{Finn+Thorne-2000},
            \begin{align}
                \label{eq:strain_lum}
                \hsn^2 & = \frac{G}{c^3} \, \scale[2]{2}{n} \, \frac{\lgwn}{\lr[2]{2 \pi \, \frstorb} \, \comdist^2}, \\
                    & = \frac{32}{5 c^8} \, \scale[2]{2}{n} \, \frac{\left(G\mchirp\right)^{10/3}}{\comdist^2} \lr[4/3]{2\pi\frstorb} \cdot g(n,e).
            \end{align}

            The GW strain from a single source, over all harmonics, can be written as \citep[][Eq.~9]{amaro-seoane2010},
            \begin{equation}
                \label{eq:gw_strain_instant}
                \hs^2(\frstgw) = \sum_{n=1}^\infty \hsn^2 = \sum_{n=1}^\infty \hscirc^2(f_p) \left(\frac{2}{n}\right)^2 \, g(n,e) \bigg|_{f_p = \frstgw/n}.
            \end{equation}
            Here, the GW strain from a circular binary is \citep[][Eq.~7]{sesana2008} (sky and polarization averaged; equivalent to Eq.~\ref{eq:strain_lum}),
            \begin{equation}
            \begin{split}
                \label{eq:gw_strain_amp_circ}
                \hscirc
                    & = \frac{8}{10^{1/2}} \frac{\left(G\mchirp\right)^{5/3}}{c^4 \, \comdist} \left(2 \pi \frstorb \right)^{2/3}, \\
                    & = \frac{8}{10^{1/2}} \frac{\left(G\mchirp_o\right)^{5/3}}{c^4 \, \lumdist} \left(2 \pi \fobsorb \right)^{2/3},
            \end{split}
            \end{equation}
            for a comoving distance $\comdist$, and luminosity distance $\lumdist$.  Note that the second version (using luminosity distance), also uses an \textit{observed} chirp-mass and observer-frame frequency (see \secref{sec:app_redshift}).

            \subsubsection{Characteristic Strain \NOTE{NEEDS REVIEW}}
            The \textbf{characteristic strain} models the effective GW amplitude when considering signal-to-noise and detectability.  For periodic signals, power accumulates proportionally to the number of cycles, such that $\hc^2 = n_\trt{cycles} \cdot \hs^2$.  There are two regimes to consider.  If the binary is very slowly evolving, $\lr{d\fobsgw/dt_o} \cdot \tdur \ll \Delta f_o$ (i.e.~the change in frequency over the course of observations is very small compared to the frequency bin), then the number of cycles is,
            \begin{equation}
                n_\trt{cycles,slow} = \fobsgw \cdot \tdur.
            \end{equation}
            If the binary is quickly evolving, then the number of cycles while the binary is emitting in the frequency bin of interest is what's relevant, i.e.
            \begin{equation}
                n_\trt{cycles,fast} = \fobsgw \cdot T_\trt{band} \msp s.t. \msp T_\trt{band} \equiv \scale{\Delta f_o }{ d\fobsgw / dt_o }.
            \end{equation}
            Note that the term in parentheses is \textit{not} the same as the GW hardening time-scale, because the frequency bin-size may be very different from the binary frequency (i.e., in general, $\Delta f_o \neq \fobsgw$).  Note also that in this latter case, the characteristic strain is not well defined \textit{at} a particular frequency, but is instead defined \textit{over} a frequency bin.  Typically, for frequency bins determined by Nyquist sampling, $\Delta f_o = 1 / \tdur$.  Putting these together we can write,
            \begin{equation}
                \hc^2(f_o,\Delta f_o|\tdur) = \hs^2(f_o) \cdot \lrs{\fobsgw \cdot \min\!\lr{\tdur, \, T_\trt{band}}}.
            \end{equation}



    % ===============================
    % ====    GWB Calculation    ====
    % ===============================

        \subsection{Gravitational Wave Background Calculation}
            \label{sec:gw_gwb_calc}

            It was shown by \citet[][Eq.5]{Phinney-2001} that the GWB spectrum, can be expressed as:
            \begin{subequations}
            \begin{align}
                % \label{eq:}
                h_c^2(f)
                & = \frac{4G}{\pi c^2 f^2} \int dz \; \frac{dn}{dz} \, \frac{1}{1+z} \scale{dE_\trt{GW}}{d \ln f_r} \bigg|_{f_r=f\lr{1+z}} \\
                & = \frac{4G}{\pi c^2 f} \int dz \; \frac{dn}{dz} \, \scale{dE_\trt{GW}}{d f_r} \bigg|_{f_r=f\lr{1+z}}.
            \end{align}
            \end{subequations}

            Following \citet[][Eq.~3.5/10/11; or Eq.~\ref{eq:gw_energy_spectrum}~\&~\ref{eq:lum_gw} above]{enoki2007a}:
            \begin{equation}
                \label{eq:strain_spectrum_en07}
                \hc^2(f) = \frac{4G}{\pi c^2 f} \int dz \, \frac{dn}{dz} \,
                    \sum_{n=1}^\infty \left[ \lgwn \cdot \frac{\thard(f_p)}{n \, f_p} \right] \bigg|_{f_p = f(1+z)/n}.
            \end{equation}

            From \citet[][Eq.~2.1]{Finn+Thorne-2000}, we can relate the luminosity to the strain amplitude at each harmonic,
            \begin{equation}
                \label{eq:lum_to_strain}
                \lgwn = \frac{c^3}{G} \scale[2]{n}{2} \cdot \hsn^2(\frstorb) \cdot \lr[2]{4\pi \frstorb} d_c^2,
            \end{equation}
            for a comoving distance $d_c$.

            \subsubsection{(Semi-)Analytic Calculation}

                Putting this together we have,
                \begin{equation}
                    h_c^2(f) = \int_0^\infty \!\! dz \; \frac{dn_c}{dz} \, \sum_n^\infty \, \hsn^2\lr{f_p} \, 4\pi c \, d_c^2 \cdot \lr{1+z} \cdot \thard(f_p) \, \bigg|_{f_p = f(1+z)/n}.
                \end{equation}
                If the GW strain is due to circular binaries \eqref{eq:gw_strain_amp_circ}, then the only harmonic with non-zero strain is $n=2$, such that $2 \, f_r = (1+z) f$:
                \begin{equation}
                    h_c^2(f) = \frac{256 \pi}{10 c^7} \int_0^\infty \!\! dz \; \frac{dn_c}{dz} \, \thard(f_r) \cdot \lr[10/3]{G \mathcal{M}} \, \lr[4/3]{2\pi f_r} \, \lr{1+z}.
                \end{equation}
                If the circular binaries are purely GW-driven, this can be further simplified to:
                \begin{equation}
                    h_c^2(f) = \frac{4 \pi}{3 c^2} \, \lr[-4/3]{2\pi f} \int_0^\infty \!\! dz \; \frac{dn_c}{dz}  \, \frac{\lr[5/3]{G \mathcal{M}}}{\lr[1/3]{1+z}}.
                \end{equation}
                From this expression, which assumes a smooth/continuous number-density distribution of binaries, and purely circular and GW-driven evolution, comes the power-law expression for the GW Background strain spectrum first derived in \citep[][Eq.~11]{Phinney-2001},
                \begin{equation}
                    h_c = A_0 \scale[-2/3]{f}{f_0} = A_{\pyr} \scale[-2/3]{f}{1 \, \pyr}.
                \end{equation}

                \NOTE{Compare to Soltan argument and number of remnants and their mass.}


            \subsubsection{Monte-Carlo Calculation}

                It was shown by \citet[][Eq.~6]{sesana2008} that the redshift evolution of the comoving number density of sources ($n_c \equiv dN/dV_c$) could be related to the instantaneous number of sources in a given frequency interval as,
                \begin{equation}
                    \label{eq:number_density_to_number_frequency}
                    \frac{dn}{dz} = \frac{d^2N}{dz \, dV_c} = \frac{d^2N}{dz \, d\ln f_p} \frac{d f_p}{dt_r} \frac{dt_r}{dz} \frac{dz}{dV_c}.
                \end{equation}
                This relationship is very subtle. \\

                From cosmography \citep[e.g.][]{Hogg1999},
                \begin{subequations}
                \begin{align}
                    \label{eq:cosmography}
                    \frac{dz}{dt} = H_0 \lr{1+z} E(z) \\
                    \frac{d V_c}{dz} = 4\pi \frac{c}{H_0} \frac{d_c^2}{E(z)}.
                \end{align}
                \end{subequations}

                Taking Eq.~\ref{eq:strain_spectrum_en07} and plugging in Eqs.~\ref{eq:lum_to_strain},~\ref{eq:number_density_to_number_frequency},~\&~\ref{eq:cosmography}, we can then write\footnote{Equivalent to \citet[][Eq.~10]{sesana2008}}:
                \begin{subequations}
                \begin{align}
                    \hc^2(f) = & \int dz \sum_n^\infty \frac{d^2 N}{dz \, d\ln f_p} \, \hsn^2(f_p) \\
                        = & \int dz \sum_n^\infty \frac{d^2 N}{dz \, d\ln f_p} \, \hscirc^2(f_p) \, \scale[2]{2}{n} \, g(n,e) \bigg|_{f_p = f_r / n = (1+z) f / n}.
                \end{align}
                \end{subequations}

        % ===== Monte-Carlo
        \subsubsection{Monte Carlo}

            From \citet[][Eq.~10]{sesana2008},
            \begin{align}
                h_c^2(f) = \int_0^\infty \!\! dz \; \frac{d^2 N}{dz \, d\ln f_r} \; h^2\lr{f_r}.
            \end{align}
            From \citet[][Eq.~6]{sesana2008} we can write,
            \begin{align}
                \frac{d^2 N}{dz \, d\ln f_r} = \frac{d n_c}{dz} \frac{dz}{dt} \frac{dt}{d\ln f_r} \frac{d V_c}{dz}.
            \end{align}

            From cosmography \citep[e.g.][]{Hogg1999},
            \begin{subequations}
            \begin{align}
                \frac{dz}{dt} = H_0 \lr{1+z} E(z) \\
                \frac{d V_c}{dz} = 4\pi \frac{c}{H_0} \frac{d_c^2}{E(z)} \\
                d_L = d_c \, (1+z)
            \end{align}
            \end{subequations}
            Putting this together we have,
            \begin{equation}
                h_c^2(f) = \int_0^\infty \!\! dz \; \frac{dn_c}{dz} \, h^2\lr{f_r} \, 4\pi c \, d_c^2 \lr{1+z} \, \frac{f_r}{df_r / dt},
            \end{equation}
            and when taking the strain to be due to a circular binary:
            \begin{equation}
                h_c^2(f) = \int_0^\infty \!\! dz \; \frac{256 \pi}{10 c^7} \frac{dn_c}{dz} \, \frac{f_r}{df_r / dt} \, \lr[10/3]{G \mathcal{M}} \, \lr[4/3]{2\pi f_r} \, \lr{1+z}.
            \end{equation}
            For circular, GW-driven binaries this can be simplified to:
            \begin{equation}
                h_c^2(f) = \int_0^\infty \!\! dz \; \frac{4 \pi}{3 c^2} \frac{dn_c}{dz}  \, \frac{\lr[5/3]{G \mathcal{M}}}{\lr[1/3]{1+z}} \, \lr[-4/3]{2\pi f_r},
            \end{equation}

            If this is being calculated from a finite volume (e.g.~simulation), then we can associate the comoving-number-density of sources at a given redshift with the number of binaries per-unit comoving volume, i.e.,
            \begin{align}
                \frac{dn_c}{dz} \, dz = N_\trt{fv}(z)/V_\trt{c,fv}.
            \end{align}


    % ====    Gravitational Wave Single Sources    ====

    \subsection{Gravitational Wave Single Sources}
    \label{sec:gw_singles_calc}

    Following \citet{taylor201505}: each polarization of the (eccentric) strain can be calculated as,

    \begin{align}
    \label{eq:}
    h_{+}(t) = & \sum_n - (1 + \cos^2 \iota) \Big[ a_n \cos(2\gamma) - b_n \sin(2\gamma)\Big] + (1+\cos^2 \iota) c_n, \nonumber \\
    h_{\times}(t) = & \sum_n 2\cos \iota \Big[a_n \sin(2\gamma) + b_n \cos(2\gamma)\Big],
    \end{align}
    where,
    \begin{align}
    \label{eq:}
    a_n & = -n\zeta \omega^{2/3}\bigg[J_{n-2}(ne) - 2eJ_{n-1}(ne) + (2/n)J_n(ne) + 2 e J_{n+1}(ne) - J_{n+2}(ne)\bigg]\cos\left[n \, l(t)\right], \nonumber \\
    b_n & = -n\zeta \omega^{2/3} (1 - e^2)^{1/2}\bigg[J_{n-2}(ne) - 2J_n (ne) + J_{n+2}(ne)\bigg]\sin\left[n \, l(t)\right], \\
    c_n & = \, 2 \zeta \omega^{2/3} J_n (ne)\cos\left[n \, l (t)\right]. \nonumber
    \end{align}
    Here the amplitude parameter is $\zeta = \mchirp^{5/3} / d_L$, $\omega = 2 \pi F_o$---where the (observed) \textbf{Keplerian mean orbital frequency} is $F_o$---and the \textbf{mean anomaly} is,
    \begin{equation}
    l(t) = l_0 + 2\pi \int_{t_0}^{t} F_o(t') dt',
    \end{equation}
    where $l_0$ is the mean anomaly at $t_0$.  The azimuthal angle $\gamma$ is measures the direction of pericenter relative to,
    \begin{equation}
    \hat{x} \equiv \frac{\hat{\Omega} + \hat{L} \cos \iota}{\left(1 - \cos^2 \iota\right)^{1/2}},
    \end{equation}
    and the \textbf{binary orbital inclination angle} is defined as,
    \begin{equation}
    \cos \iota = - \hat{L} \cdot \hat{\Omega}.
    \end{equation}

    Consider a unit vector pointing from the binary to the observer $\hat{\Omega}$.  The binary system is oriented relative to a right-handed basis triad $\{\hat{n}, \hat{p}, \hat{q}\}$, s.t.

    \begin{align}
    \hat{n} & = - \hat{\Omega}, \nonumber \\
    & = \{\sin \theta \cos \phi,\msp  \sin\theta \sin \phi, \msp \cos\theta  \} \nonumber \\
    \hat{p} & = \frac{\hat{n} \times \hat{L}}{\left| \hat{n} \times \hat{L}\right|} \\
    & = \{\cos\psi\cos\theta\cos\phi - \sin\psi\sin\phi , \msp
    \cos\psi\cos\theta\sin\phi + \sin\psi\cos\phi, \msp
    -\cos\psi\sin\theta\}, \nonumber \\
    \hat{q} & = \hat{p} \times \hat{n} \nonumber \\
    & = \{\sin\psi \cos\theta \cos\phi + \cos\psi \sin\phi,
    \msp \sin\psi \cos\theta \sin\phi - \cos\psi \cos\phi,
    \msp -\sin\psi \sin\theta \}. \nonumber
    \end{align}

    The spherical polar angles are $\theta = (\pi/2 - \mathrm{declination})$---the azimuthal angle, and $\phi = \mathrm{R.A.}$---the angle within the plane.  The angle $\psi$ is the angle between $\hat{p}$ and a line of constant $\phi$ (RA) when the binary is seen by the observer, at the origin. \\

    The binary basis vectors $\hat{p}$, $\hat{q}$ are perpendicular to the direction between the observer and the binary (and thus to the direction of observed GW propagation).  They can define basis tensors,
    \begin{align}
    e^{+}_{ab} & = \hat{p}_a \hat{p}_b - \hat{q}_a \hat{q}_b, \nonumber \\
    e^{\times}_{ab} & = \hat{p}_a \hat{q}_b + \hat{q}_a \hat{p}_b.
    \end{align}
    The GW-tensor can be written in the transverse-traceless gauge then as,
    \begin{equation}
    h_{ab}(t, \hat{\Omega}) = h_{+}(t) \, e^{+}_{ab}(\hat{\Omega}) + h_{\times}(t) \, e^{\times}_{ab}(\hat{\Omega}).
    \end{equation}

    % =================================
    % ====    Detector Response    ====
    % =================================

    \section{Detector Response}
    \label{sec:det_resp}
    Detector output $s(t)$ contains both noise and signal, i.e.~$s(t) = n(t) + h(t)$.

    % ====    Strain    ====
    \subsection{Strain}
    The \textbf{Characteristic Strain} $h_c$ is designed to include the effect of integrating an inspiralling signal over time, which builds up the Signal-to-Noise Ratio (SNR).
    \begin{equation}
    \label{eq:strain_char}
    \left[h_c(f)\right]^2 = 4 \, f^2\, \left| \tilde{h}(f)\right|^2
    \end{equation}
    where $\tilde{h}(f)$ is the Fourier transform of the strain,
    \begin{equation}
    \tilde{h}(f) = \mathcal{F}\{ h(t) \}(f) = \int_{-\infty}^\infty h(t) \,  \exp\left(-2\pi \, i f t\right).
    \end{equation}
    The characteristic-strain can be related to individual source strain as,
    \begin{equation}
    h_c^2 = \frac{\sum_k h_k^2 \, f_k}{\Delta f},
    \end{equation}
    where the summation is over all sources within a frequency bin of width $\Delta f = 1/T$.
    The noise counterpart is given by \citep{moore2015},
    \begin{equation}
    \left[h_n(f)\right]^2 = f \, S_n(f).
    \end{equation}
    The resulting SNR is,
    \begin{equation}
    \rho^2 = \int_{-\infty}^{\infty} \left[ \frac{h_c(f)}{h_n(f)} \right]^2 \, d\left(\log f\right)
    \end{equation}
    In a log-log plot, the area between an $h_c(f)$ and $h_n(f)$ curve is thus a measure of the SNR.  Eq.~\ref{eq:strain_char} does *not* work for a monochromatic source, however, instead it should be replaced by the wave amplitude times the square-root of the number of periods observed, i.e.
    \begin{equation}
    \left[h_{c,mono}(f)\right]^2 \approx N \left|\tilde{h}_{mono}(f)\right|^2,
    \end{equation}
    where $N = f^2 / \dot{f}$.

    % ====    Pulsar Timing Arrays    ====
    \subsection{Pulsar Timing Arrays}

    Order of magnitude strain sensitivity can be calculated as \citep[][Eq.~21]{rajagopal1995},

    \begin{equation}
    h \sim \frac{\sigma}{\tgw} \left( N_\mathrm{obs} T_\mathrm{obs} \right)^{-1/2} = \frac{\sigma}{\tgw} \left( \frac{T_\mathrm{obs}}{\Delta t} \right)^{-1/2}
    \end{equation}

    where $\sigma$ is the characterisitic arrival time measurement accuracy, $\tgw$ is the GW period, $N_\mathrm{obs}$ is the number of TOA measured per-year, and the $T_\mathrm{obs}$ is the total observing baseline in years.

    The timing residuals from a particular GW source can be calculated as \citep[][Eq.~20]{sesana2009},
    \begin{equation}
    \delta t_\textrm{\tiny GW} = \frac{8}{15} \alpha(f) \left(f T_\textrm{obs}\right)^{1/2},
    \end{equation}
    where the prefactor comes from the angle average of source inclinations, $f$ is the GW frequency, $T_\textrm{obs}$ is the total observing time, and $\alpha(f)$ is the characteristic amplitude of the GW induced residuals given as,
    \begin{equation}
    \alpha(f) = \left(\frac{G \mchirp}{c^3}\right)^{5/3} \frac{c}{d_L} \left( \pi f \right)^{-1/3}
    \end{equation}

    % =============   Detection Formalism   ============
    \subsection{Detection Formalism}
    Following \citep{rosado2015}.  Consider a \textbf{False Alarm Probability (FAP)} $\alpha$ with a threshold value $\alpha_0 = 0.001$, and a \textbf{Detection Probability (DP)} $\gamma$ with a threshold value of $\gamma_0 = 0.95$.

    The \textbf{Hellings \& Downs (1983) Curve ("Overlap Reduction Function")} is given by,
    \begin{equation}
    \Gamma_{ij} = \frac{3}{2} \gamma_{ij} \, \ln \left(\gamma_{ij}\right) - \frac{1}{4}\gamma_{ij} + \frac{1}{2} + \frac{1}{2}\delta_{ij}
    \end{equation}
    where
    \begin{equation}
    \gamma_{ij} = \frac{1}{2}\left[1 - \cos(\theta_{ij})\right],
    \end{equation}
    where the angle between pulsars $i$ and $j$ is $\theta_{ij}$.

    \textbf{Spectral Energy Density} for a characteristic strain $h_c$ is,
    \begin{equation}
    S_h = \frac{h_c^2}{12 \pi^2 f^3}.
    \end{equation}
    The \textit{expected power spectral density is also required}, $S_{h0}$.

    Each pulsar has a \textbf{Noise Power Spectrum} $P_i$, assuming this is described by a Gaussian random process with RMS $\sigma_i$ then,
    \begin{equation}
    P_i = 2\sigma_i^2 \Delta t
    \end{equation}

    % ==== A-Statistic ====
    \subsubsection{A-Statistic}
    Maximize $S/N_A = \mu_1 / \sigma_0$.

    \noindent \textbf{Expectation Value} of The Statistic In the Presence of a Signal
    \begin{equation}
    \mu_{1A} = 2 \pulsarsum \frac{\Gamma_{ij}^2 \, S_h \, S_{h0}}{P_i \, P_j}
    \end{equation}

    \noindent Variance of The Statistic In the \textbf{Absence} of a Signal
    \begin{equation}
    \sigma_{0A}^2 = 2 \pulsarsum \frac{\Gamma_{ij}^2 \, S_{h0}^2}{P_i \, P_j}
    \end{equation}

    \noindent Variance of The Statistic In the \textbf{Presence} of a Signal
    \begin{equation}
    \sigma_{1A}^2 = 2 \pulsarsum \frac{\Gamma_{ij}^2 \, S_{h0}^2}{\lr{P_i \, P_j}^2} \left[(P_i + S_h)(P_j + S_h) + \Gamma_{ij}^2 S_h^2\right]
    \end{equation}

    \noindent \textbf{Detection-Probability A}
    \begin{equation}
    \gamma_A
    \end{equation}

    \noindent \textbf{Signal-to-Noise Threshold A}
    \begin{equation}
    \label{eq:snr_thresh_a}
    \snr_A^T = \sqrt{2} \left[ \erfcinv{2\alpha_0} - \frac{\sigma_1}{\sigma_0} \erfcinv{2\gamma_0} \right]
    \end{equation}

    \noindent \textbf{Signal-to-Noise A}
    \begin{equation}
    \label{eq:snr_a}
    \snr_A^2 = 2 \pulsarsum \frac{\Gamma_{ij}^2 \, S_h^2}{\lr{P_i P_j}^2}
    \end{equation}

    % ==== B-Statistic ====
    \subsubsection{B-Statistic}
    Maximize $S/N_A = \mu_1 / \sigma_1$.

    \noindent \textbf{Expectation Value} of The Statistic In the Presence of a Signal
    \begin{equation}
    \mu_{1B} = 2 \pulsarsum \frac{\Gamma_{ij}^2 \, S_h \, S_{h0}}{\lr{P_i + S_{h0}} \lr{P_j + S_{h0}} + \Gamma_{ij}^2 \, S_{h0}^2}
    \end{equation}

    \noindent Variance of The Statistic In the \textbf{Absence} of a Signal
    \begin{equation}
    \sigma_{0B}^2 = 2 \pulsarsum \frac{\Gamma_{ij}^2 \, S_{h0}^2 P_i \, P_j}{\left[{\lr{P_i + S_{h0}} \lr{P_j + S_{h0}} + \Gamma_{ij}^2 \, S_{h0}^2}\right]^2}
    \end{equation}

    \noindent Variance of The Statistic In the \textbf{Presence} of a Signal
    \begin{equation}
    \sigma_{1B}^2 = 2 \pulsarsum \frac{\Gamma_{ij}^2 \, S_{h0}^2 \left[ \lr{P_i + S_h} \lr{P_j + S_h} + \Gamma_{ij}^2 \, S_h^2\right]}{\left[{\lr{P_i + S_{h0}} \lr{P_j + S_{h0}} + \Gamma_{ij}^2 \, S_{h0}^2}\right]^2}
    \end{equation}

    \noindent \textbf{Detection-Probability B}
    \begin{equation}
    \gamma_B = \frac{1}{2} \erfc{ \frac{\sqrt{2} \, \sigma_0 \erfcinv{2\alpha_0} - \mu_1}{\sqrt{2} \, \sigma_1} }
    \end{equation}

    \noindent \textbf{Signal-to-Noise Threshold B}
    \begin{equation}
    \label{eq:snr_thresh_b}
    \snr_B^T = \sqrt{2} \left[ \frac{\sigma_0}{\sigma_1} \erfcinv{2\alpha_0} - \erfcinv{2\gamma_0} \right]
    \end{equation}

    \noindent \textbf{Signal-to-Noise B}
    \begin{equation}
    \label{eq:snr_b}
    \snr_B^2 = 2 \pulsarsum \frac{\Gamma_{ij}^2 \, S_h^2}{P_i P_j + S_h\lr{P_i + P_j} + S_h^2\lr{1 + \Gamma_{ij}^2}}
    \end{equation}


    \begin{table*}\centering
    \renewcommand{\arraystretch}{1.2}
    \setlength{\tabcolsep}{4pt}
    \begin{tabular}{@{}lcc@{}}
    % \toprule
    Name 					& A-Statistic ($S/N_A = \mu_1 / \sigma_0$)		& B-Statistic ($S/N_A = \mu_1 / \sigma_1$)		\\ % \midrule
    Detection Probability	&												& $\gamma_B = \frac{1}{2} \erfc{ \frac{\sqrt{2} \, \sigma_0 \erfcinv{2\alpha_0} - \mu_1}{\sqrt{2} \, \sigma_1} }$												\\
    % \bottomrule
    \end{tabular}
    \caption{}
    \label{tab:}
    \end{table*}

    % =============   Sensitivity Curves with SNR  ============
    \subsection{Sensitivity Curves using the Signal-to-Noise Ratio}
    The `measured' SNR from a given signal power-spectral-density (PSD; $S_h$) can be calculated using Eqs.~\ref{eq:snr_a}~\&~\ref{eq:snr_b}.  By taking $S_h \rightarrow S_h^T$, and setting $\snr \equiv SNR = SNR^T$ as calculated from Eqs.~\ref{eq:snr_thresh_a}~\&~\ref{eq:snr_thresh_b}, we can calculate the `threshold' PSD as,
    \begin{equation}
    S_{h,A}^T = \snrta \frac{P_i \, P_j}{2 \Gamma_{ij}}
    \end{equation}
    and
    \begin{equation}
    S_{h,B}^T = \frac{
    \snrtb^2 \left(P_i + P_j\right) \pm \left[ \snrtb^4 \left(P_i + P_j\right)^2 + 4\snrtb^2 P_i P_j \left( \Gij^2 \left[2 - \snrtb^2\right] - \snrtb^2\right)\right]^{1/2}
    }{
    2\left[\Gij^2\left(2 - \snrtb^2\right) - \snrtb^2\right]
    }
    \end{equation}

    % ==================================
    % ====  MBH Binary Populations  ====
    % ==================================

    \section{Binary Populations}

    Define the (comoving-) number density of sources as, $n \equiv dN / dV_c$, and the distribution function of sources as $f(M,a,z) = d^2 n(M,a,z) / dM da$.  Here $M$ is the mass of each systems, $a$ is the binary separation, and $z$ is the redshift.  Note that binary frequency $f$ could be used almost interchangeably with $a$, but that both evolve as a function of time or equivalently redshift.  This formalism can easily be extended to include other parameters, such as mass-ratio and eccentricity ($q$ and $e$).  We can write the conservation equation for binaries as of function of redshift as,
    \begin{equation}
    \label{eq:conservation}
    \diffp{f}{z} + \diffp{}{m} \lrs{f \diffp{m}{z}} + \diffp{}{a} \lrs{f \diffp{a}{z}} = S_f(m, a, z).
    \end{equation}
    Here $S_f$ is a source/sink function that can account for the creation or destruction of binaries.  Binary coalescence (i.e.~as $a\rightarrow 0$) for example, can be treated using $S_f$ as a sink term.  The formation/creation of binaries is less intuitive as it depends on the particular masses/separations (etc) that define a binary.  If `binary' includes more distant pairs of MBHs, then anytime a new MBH forms (i.e.~from early universe seed formation) there will be some distance ($a$) to the nearest MBH which could be considered a binary companion.  The situation gets more complex for higher-order multiple systems.  Likely each multiple (including an isolated MBH) should be considered as a different species with a separate continuity equation, and the source/sink terms are used to convert between species.  For now we consider only binaries, and assume that all binary `formation' is encapsulated in practice from binaries moving from one part of parameter space (i.e.~large separations and redshifts) to other parts of parameter space (i.e.~smaller separations and redshifts, which are detectable).

    We consider the standard \citep[e.g.~Sesana style; see][]{Chen+2019} semi-analytic model (SAM) formalism of MBH binary populations.  In this style of calculation, $f$ is determined in a region of parameter space that can be observed/estimated, and this is \textit{evolved} to find the distribution in a different region of parameter space that is of interest.  We will express the distribution function as a product of a mass-function, and a pair fraction:
    \begin{equation}
    \label{eq:dist_func}
    f(M,a,z) = \Phi(M, z) \cdot F_a(M, a, z).
    \end{equation}
    In \citet{Chen+2019}, for example, the pair fraction is measured over some (unspecified) range of separations, and the separation-dependence is suppressed, i.e.~$f(M, z) = \Phi(M, z) \cdot F(M, z)$ s.t.~$F = \int_{a_0}^{a_1} F_a \, da$.
    The evolution of the distribution function can be derived as follows.  First, from $\eqref{eq:conservation}$, we separate terms into redshift- and time- dependency:
    \begin{equation}
    \diffp{f}{z} + \diffp{t}{z}\lr{\diffp{}{m} \lrs{f \diffp{m}{t}} + \diffp{}{a} \lrs{f \diffp{a}{t}}} = S_f(m, a, z).
    \end{equation}
    We assume that the mass change of binaries is negligible, $\diffp{m}{t} = 0$, and by ignoring coalescence (in our parameter space of interest) we can set $S_f = 0$, giving us:
    \begin{equation}
    \diffp{f}{z} = - \diffp{t}{z} \diffp{}{a} \lrs{f \diffp{a}{t}}.
    \end{equation}
    Now we plug-in \eqref{eq:dist_func} to the R.H.S.:
    \begin{equation}
    \diffp{f}{z} = - \diffp{t}{z} \Phi(M,z) \diffp{}{a} \lrs{F(M,a,z) \diffp{a}{t}}.
    \end{equation}

    The binary population is assumed to be changing only in separation and redshift, which are related by $\partial a / \partial z = (\partial a / \partial t) (\partial t / \partial z)$.  Because the overall number-density is conserved, we can take a finite step in separation and time/redshift, $a\rightarrow a'$ and $z\rightarrow z'$.  Here the time it takes for a binary to go from $a \rightarrow a'$ is $\tau(M,a,z|a')$ which leads to a redshift at the later time of $z' = z'(t + \tau)$.  The standard expression \citep[e.g.~][~Eq.~5]{Chen+2019}, requires making the approximation that,
    \begin{equation}
    \diffp{}{a} \lrs{F(M,a,z) \diffp{a}{t}} \approx \frac{F}{\Delta t} = \frac{F}{\tau(M,a,z|a')}.
    \end{equation}
    \note{Does this require assuming $\diffp{F}{a} = 0$ ??}
    Thus we finally get,
    \begin{equation}
    \diffp{f(M,a',z')}{{z'}} = \diffp{n}{{M}{a'}{z'}} = - \diffp{t}{z} \frac{\Phi(M,z) \, F(M,a,z)}{\tau(M,a,z|a')}.
    \end{equation}


    \appendix

    \section{Redshifting of GW Quantities}
    \label{sec:app_redshift}

        \subsection{Chirp Mass}

            We can define the `chirp' timescale as $t_\trt{chirp} \equiv f / \dot{f}$.  Note that $f_r = f_o \cdot (1+z)$, and $\Delta t_o = \Delta t_r \cdot (1+z)$, and thus that $\left| df / dt \right|_r = \left| df / dt \right|_o \lr[2]{1+z}$.  Unsurprisingly, then, $t_\trt{chirp,o} = t_\trt{chirp,r} \cdot \lr{1+z}$.  For GW-driven evolution, we can write the expression for the chirp time as \citep[][Eq.3]{DOrazio+Loeb-2021},
            \begin{equation}
                \label{eq:time_chirp}
                t_\trt{chirp,r} \equiv \frac{f_r}{\dot{f_r}} = \frac{5}{96} \scale[-5/3]{G \mchirp}{c^3} \lr[-8/3]{2\pi \frstorb}.
            \end{equation}
            This is a measurable time-interval, and thus we can related the observer- and rest- frames as, $t_\trt{chirp,o} = t_\trt{chirp,r} \lr{1+z}$.  If we take \eqref{eq:time_chirp} as the \textit{definition} of chirp mass as a measured quantity\footnote{i.e.~do not think of it as an intrinsic property of a binary}, we can then use \eqref{eq:time_chirp} (along with, $f_r = f_o \cdot \lrs{1+z}$) to show that,
            \begin{equation}
                \label{eq:mchirp_redshift}
                \mathcal{M}_o = \mathcal{M}_r \cdot \lr{1+z}.
            \end{equation}

        \subsection{GW Strain Amplitude}

            The relationship between rest-frame strain amplitude (now written as $h_r$, for simplicity) and luminosity (here, $\dot{E}$ for convenience), is given in \eqref{eq:strain_lum}:
            \begin{equation}
                h_r^2 = k \, \frac{\dot{E}_r}{d_c^2 \, f_r^2}.
            \end{equation}
            If we write $\dot{E} = dE/dt$, we can more clearly see that $\dot{E}_r = \dot{E}_o \, \lr[2]{1+z}$, where there is one factor of redshift for time-dilation, and one-factor for energy (e.g. redshifting of photons/gravitons).  Combined with $f_r = f_o \, \lr{1+z}$, we can then write,
            \begin{equation}
                h_r^2 = k \, \frac{\dot{E}_r}{d_c^2 \, f_r^2} = k \, \frac{\dot{E}_o \, \lr[2]{1+z}}{d_c^2 \, f_o^2 \, \lr[2]{1+z}} = k \, \frac{\dot{E}_o}{d_c^2 \, f_o^2} = h_o^2 \msp \therefore \msp h_r = h_o,
            \end{equation}
            showing that strain amplitude is invariant (as we would expect for a dimensionless measure of distance deformation).  Note that we would get the same result starting (and ending) with $\lumdist$ instead of comoving distance, but in either case we must continue to use the same distance measure on both sides of the equation.

            Starting from the standard strain amplitude expression (i.e.~first line of Eq.~\ref{eq:gw_strain_amp_circ}) and using the redshifting of chirp-mass expression \eqref{eq:mchirp_redshift}, we can then show that the second line of Eq.~\ref{eq:gw_strain_amp_circ} naturally follows, i.e.~that,
            \begin{equation}
                \hscirc
                = \frac{8}{10^{1/2}} \frac{\left(G\mchirp\right)^{5/3}}{c^4 \, \comdist} \left(2 \pi \frstorb \right)^{2/3}
                = \frac{8}{10^{1/2}} \frac{\left(G\mchirp_o\right)^{5/3}}{c^4 \, \lumdist} \left(2 \pi \fobsorb \right)^{2/3}.
            \end{equation}


        \subsection{GW Stain-Ampliude Pre-factor}


    %\bibliographystyle{plain}
    \bibliographystyle{plainnat}
    \bibliography{gw_refs}

\end{document}
